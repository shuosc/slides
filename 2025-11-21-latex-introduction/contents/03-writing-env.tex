\section{\LaTeX~Writting Env}
\begin{frame}{Overleaf}
  \begin{itemize}
    \item 打开https://cn.overleaf.com/
    \item 登陆账户
    \item 在template区域搜索相关的模板
  \end{itemize}
  \includegraphics[width=1.1\linewidth]{pic/screenshot5.png}
\end{frame}


\begin{frame}{Overleaf}
  \begin{itemize}
    \item 选择Open template
  \end{itemize}
  \includegraphics[width=1.1\linewidth]{pic/screenshot6.png}
\end{frame}

\begin{frame}{Overleaf}
  \centering
  \includegraphics[width=1.2\linewidth]{pic/screenshot7.png}
  \par
\end{frame}
\begin{frame}{Overleaf编译选项调整}
  Sometimes,something shit happens\dots\includegraphics[width=0.03\linewidth]{pic/image.png} \\
  还记得一开始我们讲的不同引擎的区别吗? \\
  到这里去设置
  \includegraphics[width=1.2\linewidth]{pic/screenshot8.png}
  \par
\end{frame}
\begin{frame}{Overleaf编译选项调整}
  \centering
  \includegraphics[width=0.4\linewidth]{pic/screenshot9.png}
  \par
\end{frame}
\begin{frame}{Overleaf编译选项调整}
  \begin{itemize}
    \item 然后就可以正常编译了\includegraphics[width=0.03\linewidth]{pic/tmp_filei_f24.png}
    \item 虽然但是,你会发现很多项目会超过免费的20s时间
    \item 甚至我们活动的beamer幻灯片(本文档)都会超时\includegraphics[width=0.05\linewidth]{pic/image.png}
    \item 所以我们也需要更好的方案\dots
  \end{itemize}

\end{frame}

\begin{frame}{Visual Studio Code}
  \includegraphics[width=0.3\linewidth]{pic/vscode.jpg}
  \begin{itemize}
    \item Free is best
    \item Unlimited
    \item Better integrated with local environments
    \item More users,better 
    \item Visual Studio Code features usable(git,MCP management,copilot,etc.)
  \end{itemize}
\end{frame}

\begin{frame}{Visual Studio Code}
  \begin{itemize}
    \item 从插件拓展选项中可以直接安装
          \includegraphics[width=1\linewidth]{pic/workshop.png}
  \end{itemize}
\end{frame}

\begin{frame}{Visual Studio Code}
  \begin{columns}
    \begin{column}{0.3\textwidth}
      LaTex Workshop插件也有很强的功能,包括
      \begin{itemize}
        \item 自动补全
        \item 控制编译
        \item 大纲
        \item \textbf{符号速查}
      \end{itemize}
    \end{column}
    \begin{column}{0.8\textwidth}
      \centering
      \includegraphics[width=1.1\linewidth]{pic/screenshot10.png}
    \end{column}
  \end{columns}
\end{frame}

\begin{frame}[fragile]{Visual Studio Code}
  \begin{itemize}
    \footnotesize
    \item 在构建\LaTeX~项目的选项中可以选择使用的编译器类型
    \item 可以通过latexmkrc文件设置编译的选项\includegraphics[width=0.03\linewidth]{pic/tmp_filei_f24.png}

  \end{itemize}
  比如
  \begin{lstlisting}[basicstyle=\ttfamily\scriptsize]
# 使用 XeLaTeX 编译 PDF
$pdf_mode = 5;
# 启用 shell escape
$xelatex = "xelatex -shell-escape -file-line-error -interaction=nonstopmode %O %S";
# 自动运行 BibTeX
$bibtex_use = 1.5;
# 最多重试5轮
$max_repeat = 5;
# 清理常用临时文件
$clean_ext = "synctex.gz nav snm vrb";
\end{lstlisting}
  \footnotesize
  如果想深入了解,有两个网站可以参考:
  \begin{itemize}
    \scriptsize
    \item \url{https://mgeier.github.io/latexmk.html}
    \item {\small\href{https://www2.yukawa.kyoto-u.ac.jp/~koudai.sugimoto/dokuwiki/doku.php?id=latex:latexmkの設定}{https://www2.yukawa.kyoto-u.ac.jp/.../latex:latexmkの設定}}
  \end{itemize}
\end{frame}
