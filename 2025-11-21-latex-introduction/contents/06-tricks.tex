\section{Tricks\&Ideas}
\begin{frame}{Document}
  这里有一些学习资源\includegraphics[width=0.05\linewidth]{pic/tmp_filei_f24.png}
  \begin{itemize}
    \item Overleaf提供的教学\url{https://www.overleaf.com/learn}
    \item amsmath包文档的中文翻译\url{https://static.latexstudio.net/article/2019/0204/amsmath-guide-zh-cn.pdf}
  \end{itemize}
\end{frame}
\begin{frame}[fragile]{坑点注意}
\begin{itemize}
  \item 转义字符:\texttt{\# \$ \% \^{} \& \_ \{ \} \~{} \textbackslash{}}需要在前面加反斜杠,防止被错误解析
  \item 空格有多种,常用有\texttt{\textbackslash{}quad \textbackslash{}\ \textbackslash{}. \textasciitilde}
  \item 正如前文所言,有的包如果使用ptricks或者eps矢量图,使用luatex会有兼容性问题,需要XeTeX
\end{itemize}
\end{frame}

\begin{frame}{Tricks4u}
\begin{itemize}
  \item \textbf{模仿}:互联网上的\LaTeX~模板通常年久\sout{甚至不能编译},内置的example很重要,尽量不要超出原有模板,否则可能会遇到包依赖冲突的问题
  \item Beamer: \LaTeX 驱动,可以制作非常漂亮的幻灯片(包括本文),\sout{但是会遇到巨量需要加fragile关键字以及兼容性问题}
  \item 多用记下常用的数学符号以外,要有稳定的查阅语法的地点
  \item 写一点,编译一点,减少出错的空间
  \item 使用Git进行版本管理(如果你不会的话,下载文档用git clone,然后用VSCode打开;每次保存时在左侧插件处提交,就可以长期本地保存历史)
\end{itemize}
\end{frame}

\begin{frame}{拒绝\LaTeX{},学习\LaTeX{},超越\LaTeX{}}
\LaTeX 真正的闪光点:\includegraphics[width=0.05\linewidth]{pic/tmp_filei_f03.png}
\begin{itemize}
  \item \LaTeX 的声明式——可以稳定复现
  \item BibTeX 的参考文献管理和稳定的工具链
  \item 数学公式的美观而快速的处理
  \item 跨文档,模板与自动化 
\end{itemize}
所以,其实我们可以\dots
\begin{itemize}
  \item 接受所见即所得——Markdown,KaTeX,Microsoft\textsuperscript{\textregistered}Word\textsuperscript{\textregistered}等工具
  \item 尝试新兴的增量编译,用现代的\sout{原神}语言Rust,用现代编程范式构建的Typst
  \item 日常使用KaTeX
  \item 认真使用Microsoft\textsuperscript{\textregistered}  Word\textsuperscript{\textregistered}
\end{itemize}
\end{frame}
\begin{frame}
认真使用Microsoft\textsuperscript{\textregistered}  Word\textsuperscript{\textregistered}
\begin{itemize}
  \item[样式] 设置word内部的正文和各级标题,可以自动应用
  \item[引用] 通过word自带的插入-引用,可以插入自动目录、交叉引用、题注
  \item[书签] 可以自定义样式,并使用超链接
  \item[子文档]  Word可以使用子文档,类似\LaTeX 的\textbackslash{}include
  \item[题注]  Word也有自动题注编号等功能\dots
\end{itemize}
那么为什么没人用呢? \includegraphics[width=0.05\linewidth]{pic/tmp_filei_f15.png}
\begin{itemize}
  \item Word将复杂度隐藏在交互,\LaTeX 将复杂度隐藏在各种模板格式文件内\sout{总会有前人替你赤石,只不过\LaTeX 实际更直观}
\end{itemize}
\end{frame}