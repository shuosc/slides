\section{\LaTeX 启动!}

\begin{frame}{Installation on Microsoft\textsuperscript{\textregistered}\ Windows\textsuperscript{\textregistered}}
  接下来的操作请拍照保存,并主动寻求staff的协助
  \begin{enumerate}
    \item 去一个镜像站\footnote{访问https://mirrors.sjtu.edu.cn/ SHU Mirror在Under Construction},寻找/CTAN/systems/texlive/Images/目录
    \item 下载texlive.iso
    \item 双击挂载,并且按图操作
  \end{enumerate}
  \includegraphics[width=0.8\linewidth]{pic/screenshot1.png}
\end{frame}

\begin{frame}{Installation on Microsoft\textsuperscript{\textregistered}\ Windows\textsuperscript{\textregistered}}
  接下来的操作请拍照保存,并主动寻求staff的协助
  \begin{itemize}
    \item 进入Advanced设置
  \end{itemize}
  \includegraphics[width=0.6\linewidth]{pic/screenshot2.png}
\end{frame}
\begin{frame}{Installation on Microsoft\textsuperscript{\textregistered}\ Windows\textsuperscript{\textregistered}}
  接下来的操作请拍照保存,并主动寻求staff的协助
  \begin{itemize}
    \item 进入Customize
  \end{itemize}
  \includegraphics[width=0.8\linewidth]{pic/screenshot3.png}
\end{frame}
\begin{frame}{Installation on Microsoft\textsuperscript{\textregistered}\ Windows\textsuperscript{\textregistered}}
  接下来的操作请拍照保存,并主动寻求staff的协助
  \begin{itemize}
    \item 然后调整安装的包
  \end{itemize}
  \includegraphics[width=0.7\linewidth]{pic/screenshot4.png}
\end{frame}
\begin{frame}{Installation on Microsoft\textsuperscript{\textregistered} Windows\textsuperscript{\textregistered}}
  \begin{itemize}
    \item 点击确定
    \item 点击安装
    \item 等待
    \item \LaTeX 启动! \includegraphics[width=0.05\linewidth]{pic/tmp_filei_f03.png}
  \end{itemize}
\end{frame}

\begin{frame}[fragile]{Installation on Apple\textsuperscript{\textregistered} macOS\textsuperscript{\textregistered}}
  \begin{itemize}
    \item 请在terminal输入brew检查是否有相应的用法输
    \item 如果没有安装brew,安装brew\footnote{请检查我们在推文和Github发布的pdf版本自行复制},记得选择清华源
          %TBD:日后修改为使用SHU自行部署的,选择这个是因为有现成的,日后修改,而且thu sjtu版本的都是bash,现代macos都是默认zsh
          \begin{lstlisting}
/bin/zsh -c "$(curl -fsSL https://gitee.com/cunkai/HomebrewCN/raw/master/Homebrew.sh)"
  \end{lstlisting}
    \item 然后在命令行输入
          \begin{lstlisting}
    brew cask install mactex
  \end{lstlisting}
    \item \LaTeX 启动! \includegraphics[width=0.05\linewidth]{pic/tmp_filei_f03.png}
  \end{itemize}
\end{frame}
\begin{frame}{Installation on GNU/Linux}
  \begin{itemize}
    \item 如果您是长期的GNU/Linux用户,我们认为您有足够的能力可以自行安装\dots \includegraphics[width=0.03\linewidth]{pic/image.png}
    \item SHUOSC欢迎你的加入
    \item 未来会有GNU/Linux的专题活动,敬请期待\includegraphics[width=0.05\linewidth]{pic/tmp_filei_f24.png}
  \end{itemize}
\end{frame}
\begin{frame}[fragile]{Installation on GNU/Linux}
  \begin{itemize}
    \item Do It Yourself
    \item Hint: Package names: \\
          \begin{tabular}{|l l|}
            \hline
            Distro \qquad\qquad\qquad\qquad                         & 包名                  \\
            Debian/Ubuntu系                                         & texlive-full        \\
            RHEL系                                                  & texlive-scheme-full \\
            SUSE系                                                  & texlive-scheme-full \\
            Arch系\footnote{Arch系Distro以拆包少著称,但是texlive也拆了\includegraphics[width=0.03\linewidth]{pic/image.png}} & texlive-meta包\&texlive-lang包        \\
            \hline
          \end{tabular}
  \end{itemize}
\end{frame}
