\section{\LaTeX~Intro}
\begin{frame}[fragile]{\LaTeX 何意味?}

  \LaTeX 是 \dots \qquad \includegraphics[width=0.05\linewidth]{pic/tmp_filei_f15.png}
  \cite{origin}
  \begin{itemize}
    \item \TeX\  is not so far
          \begin{example}
            In Markdown.....\sout{the LLM Language}
            \begin{lstlisting}[basicstyle=\ttfamily\footnotesize]
转移概率仅依赖当前状态:
\[
P(\mathbf{d}(t+1) | \mathbf{d}(t), \mathbf{d}(t-1),...
\]
        \end{lstlisting}
          \end{example}
  \end{itemize}
\end{frame}
\begin{frame}[fragile]{\LaTeX 何意味?}

  \LaTeX 是 \dots \qquad \includegraphics[width=0.05\linewidth]{pic/tmp_filei_f15.png}
  \begin{itemize}
    \item \TeX\  is not so far
          \begin{example}
            And the perfect mathematic display \dots
            \[
              P(\mathbf{d}(t+1) | \mathbf{d}(t), \mathbf{d}(t-1), \dots) = P(\mathbf{d}(t+1) | \mathbf{d}(t))
            \]
          \end{example}
  \end{itemize}
\end{frame}
\begin{frame}[fragile]{\LaTeX 何意味?}
  \LaTeX 是 \dots \qquad \includegraphics[width=0.05\linewidth]{pic/tmp_filei_f15.png}
  \begin{itemize}
    \item \TeX\  is not so far
    \item KaTeX是你已经在用的Fomula rendering,但是Limited function\includegraphics[width=0.05\linewidth]{pic/tmp_filei_f03.png}
    \item \TeX
          \begin{itemize}
            \item \TeX 的基础是宏Macro——The input and output may be a sequence of lexical tokens or characters, or a syntax tree
            \item \TeX 的排版是工业事实标准
            \item \TeX 的设计不算现代,因为驱动的是Pascal,甚至是更加古早的SAIL(Stanford AI Language)\includegraphics[width=0.05\linewidth]{pic/image.png}
          \end{itemize}
  \end{itemize}
\end{frame}
\begin{frame}{\LaTeX 何意味?}
  でも\dots \\
  1985年,Leslie\ Lamport开发了\LaTeX\ ,好时代,来临力 \\
  \begin{table}[h]

    \footnotesize
    
    \begin{tabular}{l l l}
      \hline
      \textbf{引擎} & \textbf{中文兼容性} & \textbf{说明}                           \\
      \hline
      XeTeX       & \textbf{最佳}    & 推荐使用;\texttt{xeCJK} 稳定,与主流宏包兼容性好,极少出错 \\
      LuaTeX      & \textbf{良好}    & 面向未来;功能强,但部分旧库/复杂排版偶现兼容问题             \\
      pdfTeX      & \textbf{很差}    & 不推荐;依赖过时的 \texttt{CJK} 宏包,无法直接用系统字体   \\
      \hline
    \end{tabular}
  \end{table}
\end{frame}

\begin{frame}{\LaTeX 大战 Microsoft\textsuperscript{\textregistered}Word}
  注:术业有专攻,评价需客观 \footnote{\sout{偷自TsingHuaUniversity讲稿}}
  \begin{table}[h]
    \centering
    \begin{tabular}{c|c}
      Microsoft\textsuperscript{\textregistered}  Word & \LaTeX        \\
      \hline
      字处理工具                                            & 专业排版软件        \\
      容易上手,简单直观                                        & 容易上手          \\
      所见即所得                                            & 所见即所想,所想即所得   \\
      高级功能不易掌握                                         & 进阶难,但一般用不到    \\
      处理长文档需要丰富经验                                      & 和短文档处理基本无异    \\
      花费大量时间调格式                                        & 无需担心格式,专心作者内容 \\
      公式排版差强人意                                         & 尤其擅长公式排版      \\
      二进制格式,兼容性差                                       & 文本文件,易读、稳定    \\
      付费商业许可                                           & 自由免费使用        \\
    \end{tabular}
  \end{table}

\end{frame}
