\section{Basic Grammer}
\begin{frame}[fragile]{文件结构}
  \begin{lstlisting}[basicstyle=\ttfamily]
\documentclass[a4paper]{ctexart}
% 文档类型,如 ctexart,[]内是选项,如 a4paper
% 这里开始是导言区
\usepackage{graphicx} % 引用宏包
\graphicspath{{fig/}} % 设置图片目录
% 导言区到此为止
\begin{document}
这里开始是正文
\end{document}
  \end{lstlisting}
\end{frame}
\begin{frame}[fragile]{\LaTeX{}"命令"}
  \framesubtitle{\emph{宏} (Macro)、或者\emph{控制序列} (control sequence)}
\begin{itemize}
\item 简单命令
  \begin{itemize}
    \item \verb|\命令|\hspace{2em}
    \verb|{\songti 中国人民解放军}| ~$\Rightarrow$ {中国人民解放军}
  \item \verb|\命令[可选参数]{必选参数}|\\
\verb|\section[精简标题]{这个题目实在太长了放到目录里面不太好看}|\\
$\Rightarrow$ {1.1 \hspace{1em} 这个题目实在太长了放到目录里面不太好看}
  \end{itemize}
\item 环境
  \begin{columns}[c]
  \begin{column}{0.45\textwidth}
    \begin{lstlisting}[basicstyle=\ttfamily]
\begin{equation*}
  a^2-b^2=(a+b)(a-b)
\end{equation*}
\end{lstlisting}
\end{column}\hspace{1em}
  \begin{column}{0.45\textwidth}
$ a^2-b^2=(a+b)(a-b)$
\end{column}
  \end{columns}
\end{itemize}
\end{frame}
\begin{frame}[fragile]{\LaTeX{} 常用命令}
  \begin{exampleblock}{简单命令}
\centering
\footnotesize
  \begin{tabular}{llll}
    \cmd{chapter} & \cmd{section} & \cmd{subsection} & \cmd{paragraph} \\
    章 & 节 & 小节 & 带题头段落 \\\hline
    \cmd{centering} & \cmd{emph} & \cmd{verb} & \cmd{url} \\
   居中对齐         &  强调      & 原样输出   & 超链接 \\\hline
  \cmd{footnote} & \cmd{item} & \cmd{caption} & \cmd{includegraphics} \\
   脚注 & 列表条目 & 标题 & 插入图片 \\\hline
  \cmd{label} & \cmd{cite} & \cmd{ref} \\
  标号 & 引用参考文献 & 引用图表公式等\\\hline
  \end{tabular}
\end{exampleblock}
\end{frame}
\begin{frame}[fragile]{\LaTeX{} 常用命令}
\begin{exampleblock}{环境}
\centering
\footnotesize
\begin{tabular}{lll}
  \env{table} & \env{figure} & \env{equation}\\
  表格 & 图片 & 公式 \\\hline
  \env{itemize} & \env{enumerate} & \env{description}\\
  无编号列表 & 编号列表 & 描述 \\\hline
\end{tabular}
\end{exampleblock}
\end{frame}
%
\begin{frame}{\LaTeX{}命令举例}
\cmd{chapter}{前言}{{第 1 章\hspace{1em} 前言}} \\
\cmd{section[精简标题]}{这个题目实在太长了放到目录里面不太好看}{{1.1
  \hspace{1em} 这个题目实在太长了放到目录里面不太好看}} \\
\cmd{footnote}{脚注}{何意味?\footnote{我是可爱的脚注}}
\end{frame}
\begin{frame}[fragile]{\LaTeX{} 环境举例}
  \begin{minipage}{0.4\linewidth}
    \begin{lstlisting}[basicstyle=\ttfamily\small]
\begin{itemize}
  \item 一条
  \item 次条
  \item 这一条可以分为 ...
    \begin{itemize}
      \item 子一条
    \end{itemize}
\end{itemize}
\end{lstlisting}
  \end{minipage}\hspace{1.5cm}
  \begin{minipage}{0.4\linewidth}
\begin{itemize}
  \item 一条
  \item 次条
  \item 这一条可以分为 ...
    \begin{itemize}
      \item 子一条
    \end{itemize}
\end{itemize}
  \end{minipage}
\medskip

  \begin{minipage}{0.4\linewidth}
\begin{lstlisting}
\begin{enumerate}
  \item 一条
  \item 次条
  \item 再条
\end{enumerate}
\end{lstlisting}
  \end{minipage}\hspace{1.5cm}
  \begin{minipage}{0.4\linewidth}
\begin{enumerate}
  \item 一条
  \item 次条
  \item 再条
\end{enumerate}
  \end{minipage}
\end{frame}
%

\begin{frame}[fragile]{\LaTeX{} 数学公式}

\begin{columns}
\begin{column}{.5\textwidth}
\begin{lstlisting}[basicstyle=\ttfamily\small]
$V = \frac{4}{3}\pi r^3$

\[
  V = \frac{4}{3}\pi r^3
\]

\begin{equation}
\label{eq:vsphere}
V = \frac{4}{3}\pi r^3
\end{equation}
\end{lstlisting}
\end{column}

\begin{column}{.5\textwidth}
$V = \frac{4}{3}\pi r^3$

\[
  V = \frac{4}{3}\pi r^3
\]

\begin{equation}
\label{eq:vsphere}
V = \frac{4}{3}\pi r^3
\end{equation}
\end{column}
\end{columns}
\end{frame}
\begin{frame}{\LaTeX{} 数学公式}
  \LaTeX{} 的核心优势:高质量数学排版。
  需加载宏包:\texttt{amsmath}
  公式模式:
  \begin{itemize}
    \item 行内公式:用 \texttt{\$}\,公式\,\texttt{\$}  
    例如:\texttt{\$a\textasciicircum2 + b\textasciicircum2 = c\textasciicircum2\$}  
    显示为:$a^2 + b^2 = c^2$
    \item 行间公式(推荐):
    \texttt{\textbackslash[\, 公式 \textbackslash]} \\
    (避免使用 \texttt{\$\$},与 \texttt{amsmath} 不兼容)
    \item 自动编号:\texttt{equation} 环境 \\
              多行对齐:\texttt{align}、\texttt{gather} 等
    \item 查找符号:
    \item 终端命令:\texttt{texdoc symbols}
    \item 综合手册:\texttt{comprehensive}(S. Pakin) \\
    \url{https://ctan.org/text/comprehensive}
    \item 手绘识别:Detexify \\
    \url{http://detexify.kirelabs.org}
    \item 提示:MathType 可导出 \LaTeX{},但是不被推荐
  \end{itemize}
\end{frame}
\begin{frame}[fragile]{层次与目录生成}
    \begin{columns}
    \begin{column}{.6\textwidth}
    \begin{lstlisting}[basicstyle=\ttfamily\small]
\tableofcontents
\part{范畴基础}
\chapter{函子与自然变换}
\section{函子}
\subsection{Hom 函子}
\subsubsection{表示对象}
\paragraph{Yoneda 引理}
\subparagraph{嵌入定理}
    \end{lstlisting}
    \end{column}
    \begin{column}{.4\textwidth}
    第一部分\quad 范畴基础 \\
    第一章\quad 函子与自然变换 \\
    1. 函子 \\
    1.1 Hom 函子 \\
    1.1.1 表示对象 \\
    Yoneda 引理 \\
    嵌入定理
    \end{column}
    \end{columns}
\end{frame}
    
    \begin{frame}[fragile]{列表与枚举}
    \begin{columns}
    \begin{column}{.6\textwidth}
    \begin{lstlisting}[basicstyle=\ttfamily\small]
    \begin{enumerate}
    \item \LaTeX{} 好处都有啥
      \begin{description}
        \item[好用] 体验好才是真的好
        \item[好看] 强迫症的福音
        \item[开源] 众人拾柴火焰高
      \end{description}
    \item 还有呢?
      \begin{itemize}
        \item 好处 1
        \item 好处 2
      \end{itemize}
    \end{enumerate}
    \end{lstlisting}
    \end{column}
    \begin{column}{.4\textwidth}
    {\small
    \begin{enumerate}
    \item \LaTeX{} 好处都有啥
      \begin{description}
        \item[好用] 体验好才是真的好
        \item[好看] 治疗强迫症
        \item[开源] 众人拾柴火焰高
      \end{description}
    \item 还有呢?
      \begin{itemize}
        \item 好处 1
        \item 好处 2
      \end{itemize}
    \end{enumerate}
    }
    \end{column}
    \end{columns}
    
    \end{frame}
\begin{frame}[fragile]{交叉引用与插入插图}
      \begin{itemize}
      \item 给对象命名:图片、表格、公式等\\
      \cmd{label\{name\}}
    \item 引用对象\\
      \cmd{ref\{name\}}
      \end{itemize}
    \bigskip
    
      \begin{minipage}{0.6\linewidth}
        \begin{lstlisting}
    SHUOSC logo请参见图~\ref{fig:lib}
    \begin{figure}[htbp]
      \centering
      \includegraphics[height=.2\textheight]%
      {pic/shuosc.png}
      \caption{SHUOSC logo}
      \label{fig:lib}
    \end{figure}
    \end{lstlisting}
      \end{minipage}\hfill
      \begin{minipage}{0.3\linewidth}\centering
        {SHUOSC logo请参见图~1}\\[1em]
     \includegraphics[height=0.2\textheight]{pic/shuosc.png}\\
     {\footnotesize 图~1. SHUOSC logo}
      \end{minipage}
    \end{frame}
\begin{frame}[fragile]{交叉引用与插入表格}
      \begin{columns}
      \column{.6\textwidth}
      \begin{lstlisting}
    \begin{table}[htbp]
       \caption{编号与含义}
       \label{tab:number}
       \centering
       \begin{tabular}{cl}
         \toprule
         编号 & 含义 \\
         \midrule
         1    & 第一 \\
         2    & 第二 \\
         \bottomrule
       \end{tabular}
    \end{table}

    \end{lstlisting}
    \column{.4\textwidth}
    \centering
    {\small 表~1. 编号与含义}\\[2pt]
    \begin{tabular}{cl}\toprule
    编号 & 含义 \\\midrule
    1 & 第一\\
    2  & 第二\\\bottomrule
    \end{tabular}\\[5pt]
    
      \end{columns}
    \end{frame}

 \begin{frame}[fragile]{作图与插图}
      \begin{itemize}
        \item 外部插入
    
          \begin{itemize}
            \item Mathematica、MATLAB
            \item PowerPoint、Visio、Adobe Illustrator、Inkscape
            \item Python \texttt{Matplotlib} 库、\texttt{Plots.jl}、R、Plotly 等
            \item draw.io \url{https://draw.io/}、ProcessOn \url{https://www.processon.com/} 等在线绘图网站
          \end{itemize}
    
        \item \TeX{} 内联
    
          \begin{itemize}
            \item Asymptote
            \item \alert{\texttt{pgf}/\texttt{TikZ}、\texttt{pgfplots}}
          \end{itemize}
    
        \item 插图格式
    
          \begin{itemize}
            \item 矢量图:|.pdf|
            \item 位图:|.jpg| 或 |.png|
            \item \alert{不再推荐 \texttt{.eps}}
            \item 不(完全)支持 |.svg|、|.bmp|
          \end{itemize}
        \scriptsize
        \item 一些参考:\url{https://www.zhihu.com/question/21664179}
                        \url{https://tex.stackexchange.com/q/158668}
                        \url{https://tex.stackexchange.com/q/72930}
      \end{itemize}
    \end{frame}
 \begin{frame}[fragile]{引用和参考文献}
  \scriptsize
      \begin{columns}
        \begin{column}{0.45\textwidth}
          
          \vspace{-0.3cm}
          \begin{itemize}
            \item \LaTeX{} 中的参考文献由\emph{文献数据库}(即 |.bib| 文件)生成
            \item 向数据库添加文献条目的方法:
              \begin{itemize}
                \item 使用 Mendely、Zotero、NoteExpress 等软件导出为 |.bib| 文件
                \item 从 Google Scholar、MathSciNet、ACM DL 等在线数据库导出
                \item 手工编写条目(不推荐)
              \end{itemize}
          \end{itemize}
        \end{column}
        \begin{column}{0.55\textwidth}
          \begin{lstlisting}[basicstyle=\scriptsize]
    @article{mellinger1996laser,
      author     = {Mellinger, A and Vidal, C R and Jungen, {Ch}},
      title      = {Laser reduced fluorescence study of the carbon monoxide nd triplet Rydberg series},
      journal    = {J Chem Phys},
      year       = {1996},
      volume     = {104},
      pages      = {8913--8921},
    }
            \end{lstlisting}
        \end{column}
      \end{columns}
      \begin{itemize}
        
        \item 在正文中使用 \cmd{cite\{key1, key2\}}引用条目,并在最后使用 \cmd{bibliography\{bibfile\}} 命令打印参考文献列表
        \item BibTeX 可生成各种不同格式的引用和参考文献(需要宏包支持):APA、MLA、GB/T 7714 等
        \end{itemize}
    \end{frame}
